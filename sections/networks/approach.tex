\begin{frame}{Idea}
	\centering
	\only<1>{\exampleimg{network_ex1}}%
	\only<2>{\exampleimg{network_ex2}}%
	\only<3>{\exampleimg{network_ex3}}
	\note{
		Suppose we have to find some paths between those poinsts. 
		Then we first find this path, then we have to compute this path and finally this path.
		Now you can see that all those paths use the same subpath here. 
		This subpath is now computed three times, which can become very costly if the size of the subpath increases. 
		So the idea is to make a network, so that we only have to compute point A, B to the network.
		After this we can find the lowest weight path between the two points, where A and B connect to the network, to find the path from A to B. 
	}
\end{frame}

\begin{frame}{Approach}
	\begin{itemize}
		\item Compute the visibility map once
		\item Determine a network of occluded paths
		\item Calculate connections to the network
		\item Calculate shortest connection on the network, between the two connected points. 
	\end{itemize}
	note{
		So the approach now will be to compute the visiblity map once. 
		This is so that we can find the best occluded paths on the network, which we want to include. 
		Then we determine a network of the occluded paths that we surely want in the network. 
		When we have a good network, we can easily find a good occluded path between two points. 
		We simply find the best occluded path to the network.
		Then we find the lowest weight connection between the two points on the network. 
	}
\end{frame}



\begin{frame}{Building the network}
	\begin{itemize}
		\item Chosing how to build the network is really important
		\item Well chosen network results in high-quality paths
		\item Paths can become really expensive if the network is not well chosen.
		\item To dense networks result in high computing time for the points on the network where we connect
		\item A network that is to sparse result in high computing time for the connection between the points and the network
	\end{itemize}

	\note{
		Suppose in our example we have a network like this: 
		This would result in occluded paths, that are not really useful. 
		
	}
\end{frame}

\begin{frame}{Example networks}
	\only<1>{\exampleimg{network_fail}}%
	\only<2>{\exampleimg{network_fail2}}%
	\only<3>{\exampleimg{network_fail3}}
\end{frame}

\begin{frame}{Strategies}
	\begin{itemize}
		\item Sampling based
		\item Learning based
		\item Topology based
	\end{itemize}

	\note{
		In the paper they explained multiple ways to build the network. 
		Sampling and learning based are based on the visibility map.
		They compute a lot of occluded paths on the network. 
		Then they look which path are most used and include those paths to the network.
		When the network is not complete yet, they compute the best paths between the subpaths. 
		The difference between sampling and learning based is the following:
		Sampling based takes a number of points on the map. 
		They tried multiple strategies to determine those points. 
		Then they for each of those points they find the occluded paths to his nearest neighbors.
		Learning based also takes a number of points. Here they take random points on the border.
		Then they compute the highest occluded paths to other points on the border. 
		The occluded paths in the learning based strategy are longer and thus takes more time. 
		Topology based is based on the terrain and not on the visibility map.
		Here they try to find local maxima on the terrain and include these points in to the network. 
		Next, we'll run the learning based approach on our running example
	}
\end{frame}
