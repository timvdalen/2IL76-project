\begin{frame}{Approach}
	\begin{itemize}
		\item Compute the visibility map once
		\item Determine a network of occluded paths
		\item Only calculate connections to the network
	\end{itemize}
\end{frame}

\begin{frame}{Building the network}
	\begin{itemize}
		\item Chosing how to build the network is really important
		\item Well chosen network results in high-quality paths
		\item Paths can become really expensive if the network is not well chosen.
		\item To dense networks result in high computing time for the points on the network where we connect
		\item A network that is to sparse result in high computing time for the connection between the points and the network
	\end{itemize}

	\note{
		A network is well chosen when it closely aligns with the lowest-cost subpaths over the terrain.
	}
\end{frame}

\begin{frame}{Strategies}
	\begin{itemize}
		\item Sampling based
		\item Topology based
		\item Learning based
	\end{itemize}

	\note{
		%TODO: Small explanation about all

		Next, we'll run the learning based approach on our running example
	}
\end{frame}
