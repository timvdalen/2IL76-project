\begin{frame}{Observer}
	\begin{itemize}
		\item Entity on the terrain
		\item Visible grid points \note{Those with a clear line of sight}
	\end{itemize}
\end{frame}

\begin{frame}{In our example}
	\centering
	\only<1>{
		\exampleimg{observers}
		\note{So say these dots on the terrain are our observers. We're just going to assume that these are given}
	}%
	\only<2>{
		\exampleimg{observers_visibility}
		\note{
			The next step is to figure out, for each observer, which points on the terrain are visibile and which aren't.
			Of course, the definition of visibilty can be whatever you want it.
			Here, and in the paper we're about to get to, we use that a line from the observer to the point cannot intersect with the height function.
		}
	}%
	\only<3>{
		\exampleimg{observers_line_of_sight}
		\note{However, this isn't realistic as visibily degenerates with distance}
	}%
	\only<4>{
		\exampleimg{observers_line_of_sight_one}
		\note{So we determine, for each observer, how well they can see each point}
	}%
	\only<5>{
		\exampleimg{observers_line_of_sight_all}
		\note{
			And when aggregate that (if we sum the visibility increase from 11 to 12 observers seeing a point would be the same as 1 to 2, so that's not good),
			we end up with the visibility map
		}
	}
\end{frame}
