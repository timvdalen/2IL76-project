\begin{frame}{Continous}
	\begin{itemize}
		\only<1>{
			\item $\mathds{M}$ is a square region
			\item $h$ is a height function
			\item The graph of $h$, $\Sigma$ is a terrain
		}
		\only<2>{
			%TODO: Illustration
			\item $b$ is visible from $a$ if no point in segment $ba$ lies below $\Sigma$
			\item The visibility map of an observer $o$ is for each point in $\mathds{M}$:\\
				\[V_o(p) = \left\{
					\begin{array}{lr}
						1 & : p \textnormal{ is visible from } o,\\
						0 & : \textnormal{otherwise.}
					\end{array}
					\right.
				\]
			\item Attenuated visibility map
				\note{Takes in account that vision deteriorates}
		}
	\end{itemize}
\end{frame}

\begin{frame}{Observers}
	Given $m$ observers, the aggregated visibility map:
	\[
		V_O(p) = \sum_{i=1}^{m}{V_{o_i}(p)}
	\]
\end{frame}

\begin{frame}{Coverage map}
	\only<1> {
		The coverage map of $\Sigma$, for observers $O$:
		\[
			\omega_O(p) = 1 - \epsilon^{-cV_O(p)}
		\]
		with $c > 0$

		\note{
			Now this looks very interesting, but basically all it does is make the increase in
			visibility between 0 and 1 observers bigger than the increase in visibility
			between 10 and 12 observers seeing a point
		}
	}

	\only<2> {
		The cost of a curve $\Pi$ of length $L$ is:
		\[
			c(\Pi) = \int_0^L \omega(\Pi(t)) dt
		\]
		%TODO: Explain why this is using the integral

		Cost of the most occluded path between $a$ and $b$:
		\[
			\inf_\Pi c(\Pi)
		\]
		%TODO: Explain why you can't just use min...
	}
\end{frame}
